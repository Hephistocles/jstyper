
% type-safety-proof.tex - Proof of progress and proof of preservation for my JavaScript subset

\documentclass[a4paper]{article}

\usepackage{times}
\usepackage{amsmath,amssymb}
\usepackage{fixltx2e}


\begin{document}


\title{Proof of Type Safety}
\author{Christopher Little}
\maketitle

\section*{Progress}
	\newcommand{\typable}[2][ ]{\Gamma{}\vdash#2\, |_X#1\:C#1,\:\Gamma#1'}
	\newcommand{\typed}[2]{\Gamma{}\vdash#1: #2\,|_X\:C,\:\Gamma'}
	\newcommand{\transition}[4]{\langle{}#1,#2\rangle{}\rightarrow{}\langle{}#3,#4\rangle}
	\newcommand{\indHyp}{\Phi(\Gamma, m, X, C, \Gamma')}
	\newcommand{\indHypTwo}{\Psi(\Gamma, e, T, X, C, \Gamma')}
	\newcommand{\var}{\textbf{var}}
	\newcommand{\sub}[1]{\textsubscript{#1}}

	We must prove the following $\Phi$ for all statements in the language:
	\begin{eqnarray*}
	\lefteqn{ \indHyp =  \forall s.} \\
		& & (\,\typable{m}\,) \wedge
				(dom(\Gamma) \subseteq\ dom(s))
			\implies \\
		& & \quad \textrm{$m$ is a value} \vee
				\exists m', s'. \transition{m}{s}{m'}{s'} \\
	\end{eqnarray*}

	Consider an arbitrary store $s$. Assume that the given statement $m$ is
	typable --- i.e.\ that $\typable{m}$ and that $dom(\Gamma) \subseteq\ dom(s)$.
	Since $m$ is typable, there must be some valid type derivation. We will
	prove $\indHyp$ by induction over the structure of this derivation, by
	looking in particular at the possible last steps in the derivation.
	
	\vspace{2ex}

	\begin{description}
		\item[Case \textsc{DecTypable}] \hfill 

			$m$ must be of the form \texttt{\var\ id}, and from transition
			rule \textsc{Var1}, we have 
			$$\exists m', s'.\transition{m}{s}{m'}{s'}$$
			with $m'=\epsilon$ and $s'=s$ as witnesses. Hence we have $\indHyp$.

		\item[Case \textsc{MultiDecTypable}] \hfill

			$m$ must be of the form \texttt{\var\ vd\sub{1}, vd\sub{2}}, and from
			transition rule \textsc{Var4}, we have 
			$$\exists m', s'.\transition{m}{s}{m'}{s'}$$
			with $m' = \texttt{\var\ vd\sub{1}; \var\ vd\sub{2}}$ and $s'=s$ as 
			witnesses. Hence we have $\indHyp$.

		\item[Case \textsc{DefTypable}] \hfill
			
			$m$ must be of the form \texttt{\var\ vd = e}, and a previous step
			of the derivation has shown $\typed{e}{T_2}$. Using the induction hypothesis
			gives us that one of the following subcases must be true:

			\begin{description}
				\item[Subcase $e$ is a value] \hfill \\
					
					$m$ is of right form to have a transition according to
					rule \textsc{Var3}. We thus have 
					$$\exists m', s'.\transition{m}{s}{m'}{s'}$$
					with $m'=\epsilon$ and $s'=put(s,id,e)$ as witnesses ---
					and from this $\indHyp$.
					
				\item[Subcase $\exists e', s''.\transition{e}{s}{e'}{s''}$] \hfill \\

					In this case, the preconditions for rule \textsc{Var2} are
					satisfied, implying that 
					$\transition{\texttt{\var\ id=e}}{s}{\texttt{\var\ id=e'}}{s''}$
					is a valid transition. Hence we have
					$$\exists m', s'.\transition{m}{s}{m'}{s'}$$
					with $m'=\texttt{\var\ id=e'}$ and $s'=s''$ as witnesses ---
					and from this $\indHyp$.
			\end{description}

		\item[Case \textsc{SeqTypable}] \hfill
			
			$m$ must be of the form $m_1;m_2$, and a previous step of the
			derivation has shown $\typable{m_1}$. Using the induction
			hypothesis gives us that one of the following subcases must be
			true:

			\begin{description}
				\item[Subcase $m_1$ is a value] \hfill \\
					$m$ is of the right form to have a transition according to
					rule \textsc{Seq1}. We thus have 
					$$\exists m', s'.\transition{m}{s}{m'}{s'}$$
					with $m'=m_2$ and $s'=s$ as witnesses ---
					and from this \\
					$\indHyp$. \\
					
				\item[Subcase $\exists m_1', s_1'.\transition{m_1}{s}{m_1'}{s_1'}$] \hfill \\
					In this case, the preconditions for rule \textsc{Var2} are
					satisfied, implying that 
					$\transition{m_1;m_2}{s}{m_1';m_2}{s_1'}$
					is a valid transition. Hence we have
					$$\exists m', s'.\transition{m}{s}{m'}{s'}$$
					with $m'=m_1';m_2$ and $s'=s_1'$ as witnesses ---
					and from this \\
					$\indHyp$.
			\end{description}

		\item[Case \textsc{ExpTypable}] \hfill

			For $m$ to be judged typable by this rule, we must have
			established that $\typed{e}{T}$ is valid during some earlier part
			of the derivation. We cannot use the induction hypothesis directly
			in this case, so instead we must derive the desired conclusion
			from $\typed{e}{T}$. To do this, we prove the following $\Psi$ for
			all expressions in the language, again by induction over the
			structure of the type derivation for $e$, looking in particular at
			the last step of the derivation:

			\begin{eqnarray*}
			\lefteqn{ \indHypTwo =  \forall s.} \\
				& & (\,\typed{e}{T}\,) \wedge
						(dom(\Gamma) \subseteq\ dom(s))
					\implies \\
				& & \quad \textrm{$e$ is a value} \vee
						\exists{}\, e', s'. \transition{e}{s}{e'}{s'}
			\end{eqnarray*}

			\begin{description}

				\item[Cases \textsc{V\_Num}, \textsc{V\_Bool},
				\textsc{V\_String}, \textsc{V\_Undefined}, \textsc{V\_Null} and
				\textsc{V\_Skip}] \hfill 

					In each of these cases, we immediately have that $e$ is a value,
					which gives us $\indHypTwo$ as desired.

				\item[Cases \textsc{IdType} and \textsc{IdTypeUndef}] \hfill

					In both of these cases, we have $e=id$, for which a
					transition is always available by rule \textsc{Deref}, so
					we have $\indHypTwo$ as desired.
				
				\item[Case \textsc{AssignType}] \hfill

					$e$ must be of the form \texttt{\var\ id = e\sub{1}}. We
					have that $\typed{e_1}{T_1}$, and from $\Psi$ we have one
					of the following subcases:

					\begin{description}
						\item[Subcase $e_1$ is a value] \hfill

							Rule \textsc{Assign2} gives us that 
							$$\exists{}\, e', s'. \transition{e}{s}{e'}{s'}$$
							with $e'=e_1$ and $s'=put(s,id,e_1)$ as witnesses. 
							Hence $\indHypTwo$ holds.

						\item[Subcase $\exists{}\, e_1', s'. \transition{e_1}{s}{e_1'}{s_1'}$]

							In this subcase, we have the necessary
							preconditions for rule \textsc{Assign1} to hold,
							such that we conclude $$\exists{}\, e', s'.
							\transition{e}{s}{e'}{s'}$$ with
							$e'=\texttt{id=e\sub{1}'}$ and $s'=s_1'$ as
							witnesses. Finally then, $\indHypTwo$ holds.

					\end{description}

			\end{description}

			By structural induction, then, we can conclude that $\indHypTwo$
			holds for all expressions in the language. From this and the fact
			that $\typed{e}{T}$, we can derive the conclusion that 
			$$\textrm{$m$ is a value} \vee
				\exists m', s'. \transition{m}{s}{m'}{s'}$$
			and hence that $\indHyp$ holds in this case.
	\end{description}
	
	So we have proven that $\indHyp$ holds for all statments in the language,
	and thus any statement which is well typed must either have an available
	transition, or already be a value.


\section*{Preservation}
	
	We want to show that typability is preserved by all possible transitions
	in the language. We can show this by structural induction over the
	derivation of the typability judgement for some statement $m$, using the
	following induction hypothesis:

	
\end{document}
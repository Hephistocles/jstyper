% type-safety-proof.tex - Proof of progress and proof of preservation for my JavaScript subset
\begin{theorem}[Progress for expressions]\label{app:expProgress}

  If $\typed{e}{T}$ and $\Gamma \vdash (s, \theta)$ then either e is a
  value or there exist $e',s', \theta'$ such that $\transition{e}{s}{\theta}{e'}{s'}{\theta'}$

\end{theorem}

\begin{proof} \label{app:expProgressProof}

  Take 
  \begin{multline*}
  	\indHypTwo \stackrel{\text{def}}{=} \\
  	\forall s,\theta. (\Gamma \vdash (s,\theta))\implies e=\mathtt{v}\vee
  	\exists e', s',\theta'. \transition{e}{s}{\theta}{e'}{s'}{\theta'}
  \end{multline*}

  We show that for all $\Gamma$, $e$, $T$, $C$, $\Gamma'$, if $\typed{e}{T}$
  then $\indHypTwo$ is satisfied, by rule induction on the definition of the
  type judgements. Since $\Psi$ is of the form
  $\forall s,\theta.(\Gamma\vdash(s,\theta)\implies\dots$,
  consider in all cases an arbitrary $s$ and $\theta$ and assume that $\Gamma\vdash(s,\theta)$.

  \begin{case}[V\_Num, V\_Bool, V\_String, V\_Undefined, V\_Null, V\_Closure]\label{case:prog-values} 

	For each of these rules, we immediately have that $e$ is a value, which
	gives us $\indHypTwo{}$ as desired.

  \end{case}

  \begin{case}[V\_Obj]\label{case:prog-v_obj}

	$e$ is of the form $\mathtt{\{l_1: e_1,\ \dots,\ l_k: e_k\}}$, and we have $\typed{e_i}{T}$ for $1 \leq i \leq k$.
	We can use the induction hypothesis to determine that each $e_i$ must
	either be a value or reduces futher.
	\begin{subcase}
	  all $e_i$ are values. 
	  $e$ is itself a value, so $\indHypTwo$ is
	  satisfied.
  	\end{subcase}
  	\begin{subcase}
  	  some $e_j$ is further reducible,
  	  where $j$ is the smallest index such
  	  that $e_j$ is not a value. $e$ itself must be further reducible by
  	  rule \textsc{Obj}, and again $\indHypTwo$ is satisfied.
	\end{subcase}
  \end{case}

  \begin{case}[V\_Arr]\label{case:prog-v_arr}

	$e$ is of the form $\mathtt{[e_1,\ \dots,\ e_i]}$, and we have $\typed{e_i}{T}$ for $1 \leq i \leq k$.
	The proof is similar to the case for \textsc{V\_Obj}, using rule
	\textsc{Arr} in the case when some $e_j$ can reduce further.

  \end{case}

  \begin{case}[AnonFun, AnonVoid, NamedFun, NamedVoid]\label{case:prog-fun}

	$e$ is of the form $\mathtt{function(x_1,\ \dots,\ x_2)\{m\}}$ or
	$\mathtt{function\ id(x_1,\ \dots,\ x_2)\{m\}}$. A transition is thus
	immediately available using rule \textsc{Func}.

  \end{case}

  \begin{case}[IDType]\label{case:prog-idtype}

	$e$ is of the form $\mathtt{id}$. For this to be a valid type judgement, we
	must have $id \in dom(\Gamma)$. Since we have assumed that the store is
	well-typed, we have that $dom(\Gamma) \subseteq dom(s)$, and in particular
	$id \in dom(s)$.  The preconditions are thus satisfied for rule
	\textsc{Deref}, and $e$ reduces further under this rule.

  \end{case}

  \begin{case}[PropType]\label{case:prog-proptype}

	$e$ is of the form $\mathtt{e_1.l}$. For this to be a valid type
	judgement, $\typed{e_1}{T_1}$ must hold. From the induction
	hypothesis, $e_1$ must be either a value or further reducible.
	\begin{subcase}
	  $e_1$ is further reducible.
	  The precondition for rule \textsc{Prop1} is
	  satisfied, and $e$ is further reducible under this rule.
  	\end{subcase}
  	\begin{subcase}
  	  $e_1$ is a value.
  	  It must be an object with property \texttt{\{l: v\}}
	  in order to satisfy the constraint $\{\{l:T\}\succeq T_1\}$. In this case, the rule
	  \textsc{Prop2} is applicable, and $e$ is further reducible.
  	\end{subcase}
  \end{case}

  \begin{case}[ArrayType]\label{case:prog-arraytype}

	$e$ is of the form $\mathtt{e_1[e_2]}$. For this to be a valid
	judgement, $\typed{e_1}{T_1}$ and $\typed{e_2}{T_2}$ must hold. From
	the induction hypothesis, then, we have that $e_1$ and $e_2$ are both
	either values or reduce further.

	\begin{subcase}
	  $e_1$ reduces further.
	  Then the preconditions for rule
	  \textsc{ArrGet1} are satisfied, and $e$ itself reduces further under
	  this rule.
	\end{subcase}

	\begin{subcase}
	  $e_2$ reduces further. 
	  Then the preconditions for rule
	  \textsc{ArrGet2} are satisfied, and $e$ itself reduces further under
	  this rule.
  	\end{subcase}

  	\begin{subcase}
	  both $e_1$ and $e_2$ are values.
	  Then $e_1$ must be of the form 
	  $\mathtt{[v_1,\ \dots,\ v_k]}$, to satisfy the constraint $\{[T] \succeq T_1\}$,
	  and $e_2$ must be a number to satisfy the constraint $\{T_2=number\}$.
	  $e$ is then of the form $\mathtt{[v_1,\ \dots,\ v_k][n]}$, and we 
	  can use \textsc{ArrGet3} to reduce $e$ further.
  	\end{subcase}

  \end{case}

  \begin{case}[CallType]\label{case:prog-calltype}

  	$e$ is of the form $\mathtt{e_0(e_1,\ \dots,\ e_k)}$, and we have
  	$\typed{e_i}{T_i}$ for $0\leq i \leq k$. From our induction hypothesis,
  	each of these $e_i$ must either be a value or further reducible.

  	\begin{subcase}
	  $e_0$ is further reducible.
	  The precondition for rule \textsc{Call1} is satisfied, so $e$ is further
	  reducible under this rule.
  	\end{subcase}

  	\begin{subcase}
	  $e_j$ is further reducible, 
	  where $j$ is the smallest index such that $e_j$ is not a value.
	  The precondition for rule \textsc{Call2} is satisfied, so $e$ is further
	  reducible under this rule.
  	\end{subcase}

	\begin{subcase}
	  $e_i$ are all values.
	  To satisfy the constraint $\{T = \langle T_0,\ T_1,\ \dots,\ T_i \rightarrow T_k, \gamma\rangle\}$,
	  $e_0$ must be a function closure, so \textsc{CallNamed} or \textsc{CallAnon} are
	  applicable (\textsc{CallNamed} if the function is named, or \textsc{CallAnon} if
	  it is anonymous). Thus $e$ is further reducible and $\indHypTwo$ holds.
	\end{subcase}

  \end{case}

  \begin{case}[PropCallType]\label{case:prog-propcalltype}

	Similar to case \textsc{CallType}, but using rules \textsc{PropCallNamed}
	and \textsc{PropCallAnon} instead of \textsc{CallNamed} and
	\textsc{CallAnon} respectively in the final subcase.

  \end{case}

  \begin{case}[BodyType, VoidBodyType]\label{case:prog-bodytype}

	For both rules, $e$ is of the form $\mathtt{\langle @body\{m\},\ s', \theta\rangle}$, and for this
	to be a valid type judgement, there must exist some $\gamma$ such that
	$\gamma \vdash(s', \theta)$ and $\gamma\vdash m |_C\ \gamma'$.
	Using the induction hypothesis we have that $m$ is either
	a value, a return statement, or reduces further with the store $(s', \theta)$.

	\begin{subcase}
	  $m$ is a value.
	  In this case, $e$ is of the right form to apply rule \textsc{FunBody2},
	  so $e$ reduces further. 
	\end{subcase}

	\begin{subcase}
	  $m$ is a return statement.
	  In this case, $e$ is of the right form to apply rule \textsc{FunBody3} or
	  \textsc{FunBody4} (\textsc{FunBody3} if $m$ is of the form
	  $\mathtt{return\ v}$, or \textsc{FunBody4} otherwise). Thus $e$ is
	  further reducible and $\indHypTwo$ holds.
	\end{subcase}

	\begin{subcase}
	  $m$ reduces further with store s'.
	  In other words, $\exists m', s'', \theta'. \transition{m}{s'}{\theta}{m'}{s''}{\theta'}$.
	  In this case, $e$ is of the right form to apply rule \textsc{FunBody3},
	  and the preconditions are satisfied, so $e$ reduces further. 
	\end{subcase}

  \end{case}

  \begin{case}[AssignType]\label{case:prog-assigntype}

	$e$ is of the form $\mathtt{e_0 = e_1}$, and for this rule to be satisfied
	we must have that $\typed{e_0}{T_0}$ and $\typed{e_1}{T_1}$. Using the
	induction hypothesis, we have that $e_0$ and $e_1$ are both either values
	or further reducible. Syntactically, however, $e_0$ cannot be a value ---
	it can reduce no further once it has the form $\mathtt{vRef}$.

	\begin{subcase}
	  $e_0$ reduces further and is not of the form $\mathtt{vRef}$. 
	  The precondition for rule \textsc{Assign1} is thus satisfied, and $e$
	  itself reduces further.
	\end{subcase}

	\begin{subcase}
	  $e_0$ is of the form $\mathtt{vRef}$ and $e_1$ reduces further. 
	  The precondition for rule \textsc{Assign2} is thus satisfied, and $e$
	  itself reduces further.
	\end{subcase}

	\begin{subcase}
	  $e_0$ is of the form vRef and $e_1$ is a value.
	  The only rule we can use to reduce $e$ here is \textsc{Assign}. For this
	  to be possible, we need to show that $(e_0, s, \theta) \in dom(addr)$.
	  We have assumed that the store $(s,\theta)$ is well-typed, which tells us
	  that, since $\typed{e_0}{T_0}$, we can deduce that $addr(e_0,s,\theta)$
	  is indeed well-defined. Thus \textsc{Assign} is applicable, and $e$
	  reduces further.  
	\end{subcase}

  \end{case}

  \begin{case}[AssignTypeUndef]\label{case:prog-assigntypeundef}

	$e$ is of the form $\mathtt{id = e_1}$, but $id \not\in dom(\Gamma)$, so we
	cannot simply deduce that $id \in dom(s)$. Instead, we consider both cases.

	\begin{subcase}
	  $id \in dom(s)$.
	  In this case, $addr(id, s, \theta)$ is clearly well defined, and so $e$
	  can reduce under rule \textsc{Assign}.
	\end{subcase}

  	\begin{subcase}
	  $id \not\in dom(\Gamma)$. 
	  In this case, the preconditions for \textsc{AssignUndef} are satisfied,
	  and $e$ reduces further under that rule.
	\end{subcase}

  \end{case}

  \begin{case}[NumAssignType]\label{case:prog-numassigntype}

  	$e$ is of the form $\mathtt{e_0 numOp= e_1}$. The proof proceeds as for case
  	\textsc{AssignType} except in the final subcase. 

  	\begin{subcase}
	  $e_0$ is of the form vRef, and $e_1$ is a value.
	  In this case $e$ is of the right form to apply rule \textsc{AssignNum},
	  which has no preconditions. Hence $e$ reduces further under this rule.
  	\end{subcase}

  \end{case}

  \begin{case}[NumOpType, BoolOpType, CmpOpType, NumCmpOpType]\label{case:prog-binoptype}

  	$e$ is of the form $\mathtt{e_1\ op\ e_2}$. For this to be a valid
  	judgement, $\typed{e_1}{T_1}$ and $\typed{e_1}{T_1}$ must both hold. From
  	the induction hypothesis, both $e_1$ and $e_2$ must either reduce further
  	or already be values.

  	\begin{subcase}
  	  $e_1$ is further reducible.
  	  The precondition for rule \textsc{BinOp1} is
  	  satisfied, and $e$ is further reducible under this rule. Hence
  	  $\indHypTwo$ holds.
 	\end{subcase}

  	\begin{subcase}
  	  $e_2$ is further reducible.
  	  The precondition for rule \textsc{BinOp2} is
  	  satisfied, and $e$ is further reducible under this rule. Hence
  	  $\indHypTwo$ holds.
 	\end{subcase}

	\begin{subcase}
	  $e_1$ and $e_2$ are both values. 
		% To satisfy the constraints $\{T_1=number\}$
		% and $\{T_2=number\}$, $e_1$ and $e_2$ must both be numbers.
	  The precondition for rule \textsc{BinOp3} is thus satisfied, and $e$ is
	  further reducible under this rule.
	\end{subcase}
  \end{case}

  \begin{case}[NegType]\label{case:prog-negtype}

	$e$ is of the form $\mathtt{!e_1}$. For this to be a valid judgement,
	$\typed{e_1}{T_1}$ must hold, and from the induction hypothesis $e_1$ must
	either reduce further or already be a value.

	\begin{subcase}
	  $e_1$ is further reducible.
	  Then the precondition for rule \textsc{UnOp1}
	  is satisfied, and $e$ itself reduces further.
	\end{subcase}

	\begin{subcase}
	  $e_1$ is a value.
	  In order for the constraint $\{T_1=boolean\}$ to be
	  satisfied, $e_1$ must be a boolean value. Hence 
	  rule \textsc{BoolNeg} is applicable, and $e$ itself is further reducible.
	\end{subcase}
  \end{case}

  \begin{case}[PreNumType]\label{case:prog-prenumtype}

	$e$ is of the form $\mathtt{preNum\ e_1}$. For this to be a valid judgement,
	$\typed{e_1}{T_1}$ must hold, and from the induction hypothesis $e_1$ must
	either reduce further or already be a value.

	\begin{subcase}
	  $preOp=-$. 

	  \begin{subcase}
	  	$e_1$ is further reducible.
	  	Then the precondition for rule \textsc{UnOp1}
	  	is satisfied, and $e$ itself reduces further.
	  \end{subcase}

	  \begin{subcase}
	  	$e_1$ is a value.
	  	In order to satisfy the constraint $\{T_1=number\}$,
	  	$e_1$ must be a number. Rule \textsc{NumNeg} is immediately applicable
	  	and $e$ itself reduces further.
	  \end{subcase}

	\end{subcase}

	\begin{subcase}
	  $preOp\neq-$. 
	  Syntactically, $preOp$ must be $\mathtt{--}$ or $\mathtt{++}$, and $e_1$
	  must be an assignment target (i.e.\ an expression which will reduce down
	  to a value reference).

	  \begin{subcase}
		$e_1$ reduces further. 
		The precondition for rule \textsc{UnOp1} is satisfied, and $e$ itself
		reduces further.
	  \end{subcase}

	  \begin{subcase}
		$e_1$ is of the form $\mathtt{vRef}$.
		We have assumed that the store is well-typed, so since
		$\typed{vRef}{T}$, we can deduce that $addr(\mathtt{vRef}, s, \theta)$
		is well-defined, which is all that is required for rule \textsc{PreInc}
		(if $preOp = ++$) or \textsc{PreDec} (if $preOp=--$)
	  \end{subcase}

	\end{subcase}

  \end{case}

  \begin{case}[PostOpType]\label{case:prog-postoptype}

  	Similar to the subcase of \textsc{PreNumType} where $preOp\neq-$.

  \end{case}

\end{proof}

\begin{theorem}[Progress for Statements]\label{case:prog-statProgress}

  If $\typable{m}$ and $\Gamma \vdash (s, \theta)$ then either m is a
  value, a return statement, or there exist $m',s', \theta'$ such that
  $\transition{m}{s}{\theta}{m'}{s'}{\theta'}$

\end{theorem}

\begin{proof}\label{case:prog-statProgressProof}

  Take 
  \begin{multline*}
  	\indHyp \stackrel{\text{def}}{=} 
  	\forall s,\theta. (\Gamma \vdash (s,\theta))\implies \\
  	m=\mathtt{v} \vee m=\mathtt{return\ v; m'} \vee
  	\exists m', s',\theta'. \transition{m}{s}{\theta}{m'}{s'}{\theta'}
  \end{multline*}

  We show that for all $\Gamma$, $m$, $C$, $\Gamma'$, if $\typable{m}$
  then $\indHyp$ is satisfied, by rule induction on the definition of the
  type judgements. Since $\Phi$ is of the form
  $\forall s,\theta.(\Gamma\vdash(s,\theta)\implies\dots$,
  consider in all cases an arbitrary $s$ and $\theta$ and assume that $\Gamma\vdash(s,\theta)$.

  \begin{case}[ExpTypable]\label{case:prog-expTypable}

	$m$ is of the form $e$. For this to be a valid type judgement, we must have
	$\typed{e}{T}$. From Theorem~\ref{expProgress}, we know that $e$ must be a
	value or reduce further, and so $\indHyp$ is satisfied.

  \end{case}

  \begin{case}[FunDef]\label{case:prog-funDef}

	$m$ is of the form $\mathtt{@def\ function\ id(x_1, \dots, x_i)\{m\}}$. There
	is a transition immediately possible using rule \textsc{FuncDef}, so $\indHyp$
	is satisfied.

  \end{case}

  \begin{case}[RetTypable1]\label{case:prog-retTypable1}

  	$m$ is of the form $\mathtt{return}$, and there is a transition immediately
  	possible using rule \textsc{Ret2}, so $\indHyp$ is satisfied. 

  \end{case}

  \begin{case}[RetTypable2, RetTypable3, RetTypable4]\label{case:prog-retTypable}

  	$m$ is of the form $\mathtt{return\ e}$, and we have $\typed{e}{T}$. From
  	Theorem \ref{expProgress}, we know that $e$ is either a value or reduces
  	further.

  	\begin{subcase}
  	  $e$ reduces further.
  	  The preconditions are satisfied for rule \textsc{Ret1}, so $m$ reduces
  	  further and $\indHyp$ is satisfied.  
  	\end{subcase}

  	\begin{subcase}
  	  $e$ is a value.
  	  In this case, $m$ is an irreducible return statement, and $\indHyp$ is
  	  satisfied.
	\end{subcase}  

  \end{case}

  \begin{case}[SeqTypable]\label{case:prog-seqtypable}

	$m$ is of the form $m_1; \dots; m_i$, and for each $i$ we have
	$\typable{m_i}$. Using the induction hypothesis, $m_1$ in particular is
	either a value, a return statement, or reduces further.

	\begin{subcase}
	  $m_1$ is a value.
	  $m$ is of the right format to apply rule \textsc{Seq1}, and $m$ reduces
	  further. 
	\end{subcase}

	\begin{subcase}
	  $m_1$ is a return statement.
	  In this case, $m$ itself is also an irreducible return statement, so
	  $\indHyp$ is satisfied. 
	\end{subcase}

	\begin{subcase}
	  $m_1$ reduces further.
	  The preconditions are satisfied for rule \textsc{Seq2}, so $m$ reduces
	  further and $\indHyp$ is satisfied.
	\end{subcase}

  \end{case}

  \begin{case}[IfTypable]\label{case:prog-iftypable}

	$m$ is of the form $\mathtt{if\ (e)\ \{m_1\}\ else\ \{m_2\}}$, and we have
	that $\typed{e}{T}$. Using Theorem~\ref{expProgress}, we know that $e$ is
	either a value or reduces further.

	\begin{subcase}
	  $e$ reduces further. 
	  The preconditions are satisfied for \textsc{If1}, so $m$ reduces
	  further and $\indHyp$ is satsfied.
	\end{subcase}

	\begin{subcase}
	  $e$ is a value.
	  In order to satisfy the constraint $\{T_0 = boolean\}$, $e$ must be either
	  $\mathtt{true}$ or $\mathtt{false}$. If it is true, $m$ is of the right
	  form to apply rule \textsc{If2} and reduce further. If $e$ is false, $m$
	  is of the right form to reduce under rule \textsc{If3}. Either way,
	  $\indHyp$ is satisfied.
	\end{subcase}

  \end{case}

  \begin{case}[IfTypable2]\label{case:prog-iftypable2}

	$m$ is of the form $\mathtt{if\ (e)\ \{m_1\}}$. $m$ can immediately reduce
	further using rule \textsc{If4}, so $\indHyp$ is satisfied.

  \end{case}

  \begin{case}[WhileTypable, ForTypable]\label{case:prog-looptypable}

  	For both of these type judgements, $m$ is immediately of the right form to
  	reduce into an if-expression. For \textsc{WhileTypable}, the transition is
  	described by rule \textsc{While}, and for \textsc{ForTypable}, the
  	transition is described by \textsc{For}. Since $m$ reduces further,
  	$\indHyp$ is satisfied.

  \end{case}

  \begin{case}[DecTypable, MultiDecTypable]\label{case:prog-dectypable}

	Similar to the case for \textsc{WhileTypable}, reducing under rules
	\textsc{Var1} and \textsc{Var4} for \textsc{DecTypable} and
	\textsc{MultiDecTypable} respectively.

  \end{case}


  \begin{case}[DefTypable]\label{case:prog-deftypable}

	$m$ is of the form $\mathtt{var\ id=e}$, where $\typed{e}{T}$. Using
	Theorem~\ref{expProgress}, $e$ is either a value or reduces further. 

	\begin{subcase}
	  $e$ reduces further.
	  The preconditions for rule \textsc{Var2} are satisfied, and $m$ reduces
	  further under this rule.
	\end{subcase}

	\begin{subcase}
	  $e$ is a value. 
	  $m$ is of the right form to reduce under rule \textsc{Var3}, and so
	\end{subcase}

  \end{case}

\end{proof}

